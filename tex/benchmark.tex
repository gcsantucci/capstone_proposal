
\section{Benchmark Model}

In this work, two numbers will be used 

\begin{equation} \label{eq:eff}
\epsilon = r (\textrm{recall}) = \frac{TP}{TP + FN},
\end{equation}

where $TP$ is the number of true positives\footnote{The number of events that were classified correctly as being in class 1.} and $FN$ is the number of false negatives\footnote{The number of events that were classified incorrectly as being in class 0.}. So efficiency is the same as recall and it measures how well the classifier is able to select events of the signal class.

The 


In this section, provide the details for a benchmark model or result that relates to the domain, problem statement, and intended solution. Ideally, the benchmark model or result contextualizes existing methods or known information in the domain and problem given, which could then be objectively compared to the solution. Describe how the benchmark model or result is measurable (can be measured by some metric and clearly observed) with thorough detail.


